\documentclass[A4paper,12pt]{article}

\textwidth=7in
\textheight=8.5in
\topmargin=-1in
\headheight=1in
\headsep=.5in
\hoffset=-.85in


\usepackage{termcal}
\usepackage{fancyhdr}

\pagestyle{fancy}


\fancyhead{}
\fancyfoot{}

\chead{\textbf{PSYC 3\\Cognitive Foundations}}
\lhead{Instructor: Bernhard Angele}
\rhead{MW 11.00 AM -- 1.50 PM\\ Warren Lecture Hall 2207}
\rfoot{\thepage}


\renewcommand{\thefootnote}{\fnsymbol{footnote}}


% Few useful commands (our classes always meet either on Monday and Wednesday 
% or on Tuesday and Thursday)

\newcommand{\MWClass}{%
	\calday[Monday]{\classday} % Monday
	\skipday % Tuesday (no class)
	\calday[Wednesday]{\classday} % Wednesday
	\skipday % Thursday (no class)
	\skipday % Friday (no class) 
	\skipday\skipday % weekend (no class)
}

\newcommand{\Holiday}[2]{%
	\options{#1}{\noclassday}
	\caltext{#1}{#2}
}

\usepackage{hyperref}

\begin{document}
	\setlength{\unitlength}{1in}
	
	\renewcommand{\arraystretch}{2}
	
	\vskip.25in
	\noindent\textbf{Instructor:}  Bernhard Angele, M.A.\\
	\textbf{Email}: \href{http://piazza.com/ucsd/summer2013/psyc3/home}{Piazza} or \href{mailto:bangele@ucsd.edu}{bangele@ucsd.edu}\\
	\noindent\textbf{Office Hours:} Wednesday @ 2.15 PM -- 4.15 PM\\
	\noindent\textbf{Location:} Mandeville Coffee Cart (Art of Espresso); Mandler Hall 3572 if raining.
	
	\vskip.25in

	\noindent\textbf{TA:}  Randy Tran, M.A.\\
	\textbf{Email}: \href{http://piazza.com/ucsd/summer2013/psyc3/home}{Piazza} or \href{mailto:r4tran@ucsd.edu}{r4tran@ucsd.edu}\\
	\noindent\textbf{Office Hours:} Monday @ 10.00 AM -- 10.50 AM \& By Appointment\\
		\noindent\textbf{Location:} McGill 1111
	
	\vskip.25in
	
	\section{Course overview and goals}This course is designed to introduce students to the broad field of cognitive psychology, the scientific study of mental processes. We will cover topics such as perception, attention, memory, language, and decision making as well as the relationship between cognitive psychology and cognitive neuroscience. Students will be introduced to the basics of the scientific method and are encouraged to come up with their own empirical questions about cognition. At the end of the class, you will have a solid understanding of basic concepts, methods, and results in the study of cognition and will be able to appreciate and critically evaluate the basic meaning, if not every detail, of novel findings in cognitive psychology and neuroscience.
	
	\section{Textbook (required)} Goldstein, E. B. (2008). Cognitive psychology: Connecting mind, research, and everyday experience (3rd edition). Wadsworth Publishing Company. ISBN-10: 0840033559\\
	\newline
	New copies of this textbook come with access to Coglab, a website on which you can try a number of cognitive psychology experiments for yourself. While this access is definitely helpful in studying, it will \textit{not} be required for this class. The loose-leaf edition of this book should be identical to the hardcover version.

	\section{Course website and discussion platform} We will use TED (\href{http://ted.ucsd.edu}{ted.ucsd.edu}) to securely post grades and lecture notes. Additionally, in order to promote peer teaching and to avoid answering the same questions repeatedly, we are going to use Piazza as a discussion platform. If you have not done so already, please join the class at \mbox{\href{http://piazza.com/ucsd/summer2013/psyc3}{piazza.com/ucsd/summer2013/psyc3}} in order to gain access to the questions and answers. Please post any general questions you have about the course content or logistics on Piazza. We will try to answer any questions (anonymous or not) as quickly as possible. You should only contact us via email for questions that are specific to your student record (e.g. your grades). If you do send us a general question via email, rather than only replying to you, we will also post it and our answer on Piazza, so that other students can also benefit from it (your name will be removed, of course). Please keep in mind the UC San Diego Principles of Community (\href{http://www.ucsd.edu/explore/about/principles.html}{http://www.ucsd.edu/explore/about/principles.html}) when posting on Piazza. Should there be violations of these principles, we will have to remove the ability to post anonymously for everyone.

	\section{Course response system:} Due to the size of the class, individual student participation and feedback can be quite difficult. Because of this, we will be using the i$>$clicker system. If you wish to earn course participation credit and do not have an i$>$clicker remote already, you will have to purchase one by July 3rd, e.g. from the UCSD bookstore. Both i$>$clicker versions (1 and 2) will work. You must register your i$>$Clicker on TED by Friday of the first week (July 5th). During each class, I will pose at least one or two questions, some of which will resemble exam questions and some of which are more difficult in order to encourage deeper thinking about the issues we will encounter during the class. You will receive credit both for answering a question and for answering it correctly, if there is a correct answer (see below for grading specifics). Typically, you will first answer the question on your own before answering it a second time after discussing it with your neighbors. We will also sometimes use the i$>$Clicker system for quick feedback on the class.

	\vspace*{.15in}

%\newpage

\section{Grade Policy} Exams and grading:
Your grade will consist of two components: exams and clicker participation credit. You can earn a total of 100 points in this class. Your letter grade will be determined based on your point score as detailed in the following table. If you take the course as no pass/pass, grades of C- and better will count as "pass".
 
 \vskip.25in

\begin{minipage}{\textwidth}
	
\centering
\begin{tabular}{|l|c|}
	\hline
	Grade & Minimum score \\ \hline
	A+ & 97 \\ \hline
	A& 93  \\ \hline
	A-& 90 \\ \hline
	B+	& 87 \\ \hline
	B	& 83 \\ \hline
	B-	& 80 \\ \hline
	C+	& 77 \\ \hline
	C	& 73 \\ \hline
	C-	& 70 \\ \hline
	D	& 63 \\ \hline
	F	& 0 \\ \hline
\end{tabular}
\end{minipage}

\vskip.25in
\textbf{Under \textit{no circumstances} will I offer extra credit to individual students, no matter what the situation. Please do not send me or the TA any requests to that effect---we will not reply to any such emails.}
\newline\newline
\textbf{I will not apply a curve or adjust grades in any way}, with one exception:  After the final exam, I may decide to increase the score of all students by a certain amount to bring the grade average in line with previous PSYC 3 courses. Adjustments will never decrease any score, except for corrections of clerical errors.

\subsection{Exams}
There will be three exams, two midterms and the final. They will not be cumulative and will not cover material from previous midterms. Each exam will have 20 multiple-choice questions, each of which will be worth two points, and two short-answer questions, each of which will be worth 3 points. No scantron is required, but you must bring a blue book, a \#2 pencil and your Student ID. Students without ID will not be able to take the exam. Please note that the answer sheets will be scanned and automatically scored. Missing, messy or incorrect marks can lead to significant delays in scoring the exams and will result in point deductions. For each student, the exam with the lowest score will be dropped, for a maximum total exam score of $20*2 + 2*3 + 20*2 + 2*3 = 92$. Please note that you can miss one exam and still get the maximum score. Because of this, no make-up exams will be given, unless you can provide a doctor's statement showing that you were unable to attend two or more of the exams for a medical reason. The exams will be held on the dates stated in this syllabus (see below).  You should be aware that university policy states that any student who arrives to class after the first exam has been turned in will not be allowed to take the exam and I will strictly follow this policy.
\subsection{Clicker participation credit}
You will receive one participation point based for questions that you respond to and one additional participation point for each question that you answer correctly. As you can see from the table below, you will receive the maximum number of points even if you fail to get 25\% of the points. In order to get credit for your clicker use, you must register your i$<$clicker with your course record on TED. Responses from unregistered i$<$clickers cannot be applied to your grade after July 5th. The maximum number of points that can be earned in this class is $23*2 + 8 = 100$. A summary of your current participation performance will be posted on TED and updated regularly. Make sure to check it frequently and notify me or the TAs of any problems or errors. Clicker credit can only be earned by using the i$<$clicker in class. Because you can miss up to 25\% of the clicker credit without it having an impact on your grade, there will be NO make-up clicker credit in any form, no matter the circumstances.

\vskip.25in
\begin{minipage}{\textwidth}
\centering
\begin{tabular}{|c|c|}
\hline
\% possible points & Credit \\ \hline
75 and up & 8 \\ \hline
62.5 and up & 7 \\ \hline
50  and up & 6 \\ \hline
37.5 and up & 5 \\ \hline
25 and up & 4 \\ \hline
12.5 and up & 3 \\ \hline
10 and up & 2 \\ \hline
5 and up & 1 \\ \hline
Below 5 & 0\\ \hline
\end{tabular}
\end{minipage}
\vskip.25in

\subsection{Extra credit}
Students may earn extra credit by participating in psychology, cognitive science, and linguistics experiments via the SONA website: \href{http://ucsd.sona-systems.com/}{http://ucsd.sona-systems.com/}. Students will earn one extra point for each credit earned on SONA and assigned to PSYC 3 up to a total of five points of extra credit. Please see the SONA handout (available on the Syllabus page in TED) for details on how to sign up and assign credit. Earning extra credit will raise a student's total score by one up to five points, but no total score will exceed 100 points. For questions regarding SONA and participation in UCSD Psychology experiments please contact Psychology Student Services Office at 1553 Mandler Hall. 
\newline
\\\textbf{Alternative assignment:} If you do not wish to participate in experiments but wish to earn extra credit, you may write an extra credit paper instead. For this paper, you will write answers to the "Think about it" questions for up to five of the chapters of the textbook. These answers should be no longer than two or three sentences per question. If your answers are satisfactory, you will receive one extra credit per chapter (up to five). This alternative assignment must be approved by me by the end of week 2 and must be turned in by Wednesday of the last week of class. You are encouraged to contact me or the TA during office hours for help with your draft.

\vskip.25in

\section{Special needs}
Students with documentable special needs should contact the instructor or the TA as well as the Psychology Student Services Office (1553 Mandler Hall, psycadvising@ucsd.edu) as soon as possible and must submit their AFA form from the Office for Students with Disabilities (OSD) two weeks before the first exam for which they require accommodations.


\section{Academic dishonesty and cheating}  Academic dishonesty and cheating of any kind will not be tolerated. Any academic integrity violations such as cheating on an exam, recording clicker responses for another student, etc. will be reported to the Academic Integrity Office. Sanctions may include the creation of a disciplinary record, disciplinary probation, suspension for up to one year, dismissal from the university, and/or an academic sanction (an F grade which cannot be replaced by retaking the class). 
See http://senate.ucsd.edu/manual/appendices/app2.htm for UC San Diego policies on academic dishonesty.
\pagebreak
\section{Course Outline:}
\begin{center}
	\begin{calendar}{7/1/2013}{5} % Semester starts on 1/11/2010 and last for 16
		% weeks, including finals week
		\setlength{\calboxdepth}{.3in}
		\MWClass
		% schedule
		\caltexton{1}{How to study for this class. What is cognitive psychology? History of cognitive psychology. Cognitive Neuroscience. \\\textit{Chapter 1 and 2, Bjork et al. (2013)\footnote{Bjork, R. A., Dunlosky, J., \& Kornell, N. (2013). Self-regulated learning: Beliefs, techniques, and illusions.\textit{ Annual Review of Psychology}, 64, 417-444. You can find a PDF of this article on TED.}}}
		\caltextnext{Perception: Visual consciousness. Pattern recognition. Modularity. Attention: Filter theories.\\ \textit{Chapter 3}}
		\caltextnext{Attention: Capacity theories and bottleneck. Visual attention. \textit{Chapter 4}}
		\caltextnext{\textbf{Midterm 1 (Chapters 1--4)}. \newline\newline \textbf{Lecture:} Memory: Sensory memory and short-term memory. \textit{Chapter 5}}
		
		\caltextnext{Memory: Working memory. Long term memory -- Structure. \textit{Chapters 5 and 6}}
		
		\caltextnext{Memory: Long-term memory -- Types, Encoding and retrieval. \textit{Chapters 6 and 7}}
		
		\caltextnext{Memory: Everyday memory. Reconstructive retrieval, source monitoring, eyewitness testimony. Knowledge representation: Concepts and Propositions. Semantic memory. \textit{Chapters 8 and 9}}
		
		\caltextnext{\textbf{Midterm 2 (Chapters 5 --9)}. \newline\newline \textbf{Lecture:} Decision making: Reasoning. Utility and heuristics. \textit{Chapter 13}}
		
		\caltextnext{Decision making (continued). Language: Structure of language. \textit{Chapters 11 and 13}}
		
		\caltextnext{Language: Comprehension and Production. Reading. \textit{Chapter 11}}
		% ... and so on
		
		% Holidays
		\Holiday{1/18/2010}{Martin Luther King Day}
		\Holiday{3/8/2010}{Spring Break}
		% ... and so on
		
	\end{calendar}
\end{center}

\noindent The \textbf{final exam} (covering Chapters 11 and 13) will take place on \textbf{Friday, August 2nd from 11.30 am to 2.29 pm}. The location is still to be determined. Please note that the dates and topics given in this syllabus are preliminary and subject to change, except for the exam dates. Check TED frequently for updated versions of this syllabus. Office hours will take place every week. Occasionally, office hours have to be canceled or relocated due to unforeseen events. Usually, we will try to offer an alternative time if this happens. If you are planning to go to office hours, please check TED beforehand to make sure they are taking place as scheduled. Individual appointments outside of office hours will be made at our discretion and may not be available during some weeks. You can review your midterm results up to the last day of class before the following midterm or the final. If you wish to see the original midterm (as opposed to only your answers or a digital version of your answer sheet), you must make an appointment with me or the TA. 

\vskip.25in
\noindent\textbf{Summary of important dates}:
\begin{center} \begin{minipage}{5in}
		\begin{flushleft}
			Deadline for i$<$clicker registration \dotfill July 5th\\
			Add Deadline \dotfill July 8th\\
			Deadline for requesting alternative extra credit \dotfill July 10th\\
			Deadline for dropping without \textit{W}\dotfill July 12th\\
			First Midterm \dotfill July 10th\\
			Second Midterm  \dotfill July 24th\\
			Deadline for extra credit requirements \dotfill July 31st\\
			Course Final \dotfill August 2nd\\
		\end{flushleft}
	\end{minipage}
\end{center}


\end{document}