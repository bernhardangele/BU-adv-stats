\documentclass[a4]{article}

%opening
\title{Quiz: Do you need this course?}
\author{From Shravan Vasishth's statistics notes}


\begin{document}

\section{Quiz: Do you need this course?}
From Shravan Vasishth's statistics notes.

\textbf{Instructions}: choose only one answer by circling the relevant letter. If you don't know the answer, just leave the answer blank. 

\begin{enumerate}
\item Standard error is

\begin{enumerate}
\item[a] the standard deviation of the sample scores
\item[b] the standard deviation of the distribution of sample means
\item[c] the square root of the sample variance
\item[d] 2 times the standard deviation of sample scores
\end{enumerate}

\item
If we sum up the differences of each sample score from the sample's mean (average) we will always get

\begin{enumerate}
\item[a] a large number
\item[b] the number zero
\item[c] a different number each time, sometimes large, sometimes small
\item[d] the number one
\end{enumerate}

\item As sample size increases, the standard error of the sample should

\begin{enumerate}
\item[a]
increase
\item[b]
decrease
\item[c]
remain unchanged
\end{enumerate}

\item
The 95\% confidence interval tells you

\begin{enumerate}
\item[a]
that the probability is 95\% that the population mean is equal to the sample mean
\item[b]
that the sample mean lies within this interval with probability 95\%
\item[c]
that the population mean lies within this interval with probability 95\%
\item[d] 
none of the above
\end{enumerate}

\item
The 95\% confidence interval is roughly equal to 

\begin{enumerate}
\item[a]
 0.5 times the standard error
\item[b]
 1 times the standard error
\item[c]
1.5 times the standard error
\item[d]
2 times the standard error
\end{enumerate}

\item
The 95\% confidence interval is --- the 90\% confidence interval

\begin{enumerate}
\item[a]
wider than
\item[b]
narrower than
\item[c]
same as
\end{enumerate}

\item
A p-value is

\begin{enumerate}
\item[a]
the probability of the null hypothesis being true
\item[b]
the probability of the null hypothesis being false
\item[c]
the probability of the alternative hypothesis being true
\item[d]
the probability of getting the sample mean that you got (or a value more extreme) assuming the null hypothesis is true
\item[e]
the probability of getting the sample mean that you got (or a value less extreme) assuming the null hypothesis is true
\end{enumerate}

\item
If Type I error probability, alpha, is 0.05 in a t-test, then

\begin{enumerate}
\item[a]
we have a 5\% probability of rejecting the null hypothesis when it is actually true
\item[b]
we have a 95\% probability of rejecting the null hypothesis when it is actually true
\item[c]
we necessarily have low power
\item[d]
we necessarily have high power
\end{enumerate}

\item
Type II error probability is

\begin{enumerate}
\item[a]
the probability of accepting the null when it's true
\item[b]
the probability of accepting the null when it's false
\item[c]
the probability of rejecting the null when it's true
\item[d]
the probability of rejecting the null when it's false
\end{enumerate}

\item
When power increases
\begin{enumerate}
\item[a]
Type II error probability decreases
\item[b]
Type II error probability increases
\item[c]
Type II error probability remains unchanged
\end{enumerate}

\item
If we compare two means from two samples, and the p$>$0.05 (p is greater than 0.05), we can conclude 

\begin{enumerate}
\item[a]
that the two samples comes from two populations with different means
\item[b]
 that the two samples comes from two populations with identical means
\item[c]
that we don't know whether two samples comes from two populations with identical means or not
\end{enumerate}

\end{enumerate}


\end{document}
